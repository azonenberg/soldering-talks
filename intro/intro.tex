\documentclass{beamer}
\mode<presentation>
\usetheme{Berlin}

\title{Introduction to Soldering}
\author{Andrew D. Zonenberg}
\institute{Antikernel Labs}
\date{\today}

\begin{document}

\frame{\titlepage}

\begin{frame}
\frametitle{About Me}

\begin{itemize}
\item 2013-2015: Process Engineer, LIB3
\item 2015: Ph.D Computer Science (RPI)
\item 2015 - present: Sr. Security Consultant, IOActive
\item 2016 - present: Research Scientist, Antikernel Labs
\item Designing and soldering PCBs since 2009
\end{itemize}
\end{frame}

\begin{frame}
\frametitle{What is Soldering?}
\begin{itemize}
\item Joining parts with a \emph{low melting point} filler metal
\item Reversible - filler can be removed to separate the parts
\item Provides both \emph{electrical} and \emph{mechanical} connection
\end{itemize}
\end{frame}

\begin{frame}
\frametitle{Safety concerns}
\begin{itemize}
\item Toxic heavy metals (Pb, Cd, Sb)
\item High temperatures
\item Flux smoke
\end{itemize}
\end{frame}

\begin{frame}
\frametitle{Soldering Processes}
We can classify soldering processes by the heat source used.
\begin{itemize}
\item \textbf{Iron soldering} \\
A heated metal tip provides heat to the joint by \emph{conduction}
\item \textbf{Reflow soldering} \\
Non-contact heating using \emph{convection} (oven / heat gun) or \emph{radiation} (heat lamp)
\end{itemize}
\end{frame}

\begin{frame}
\frametitle{Anatomy of a Soldering Iron}
\begin{itemize}
\item Insulated handle
\item Heat source
\item Tip
\end{itemize}
\end{frame}

\begin{frame}
\frametitle{Soldering Iron Heat Sources}
\begin{itemize}
\item \textbf{Combustion} \\
Portable, but poor temperature control. Rarely used.
\item \textbf{Electrical} \\
Universally used for precision electronics work.
\end{itemize}
\end{frame}

\begin{frame}
\frametitle{Soldering Iron Temperature Control}
\begin{itemize}
\item \textbf{None} \\
Lowest cost. No temperature control whatsoever.
\item \textbf{Resistive element + sensor} \\
Most commonly used. Easily adjustable as needed.
\item \textbf{Curie point} \\
More expensive. Very stable but not adjustable.
\end{itemize}
\end{frame}

\begin{frame}
\frametitle{Solder Materials}
\begin{itemize}
\item \textbf{Sn-Pb} \\
Industry standard until early 2000s
\item \textbf{Sn-Ag-Cu (SAC)} \\
Lead free, higher melting point. Widely used.
\item \textbf{Sn-In-Bi} \\
Various alloys with low melting point. Brittle.
\end{itemize}
\end{frame}

\begin{frame}
\frametitle{Soldering Fluxes}
Fluxes
\begin{itemize}
\item
\end{itemize}
\end{frame}

\end{document}
